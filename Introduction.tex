% % % Set the style for this file:
\pagestyle{introduction}

% % % Beginning of the section
\section*{\uppercase{Introduction}}\label{intro}
	\bigskip
	\bigskip
	The new High Luminosity LHC (HL-LHC) program that is planned to begin operating from 2026, requires a serious update of the current ATLAS detector. Specially the inner tracker detector (ITk), which is the inner most layer of ATLAS, needs a redesign in order to withstand the higher radiation damage and, at the same time, provide faster response to maintain low pileup with the expected higher particles rates.
	In this work we focus on one specific part of the detector: the strip end-caps petals. Each strip end-cap consists of six rings, which individually contain 32 petals arranged radially around them. Petals are the local support structures holding the ATLAS end-cap ITk strip sensors as well as the common electrical, optical and cooling services \ref{ref1}. A series of tests for the petals are being designed to cover both Quality Assurance (QA) and Quality Control (QC) during the production phase. These include a series of tests on their assembled structure (through metrology surveys), their electrical functionality and their thermo-mechanical properties.
	Thermal test are a very important part of the R\&D process for the ATLAS ITk strip end-caps sensors. They provide the means to understand certain sensor properties like ASIC burn in, thermal cycling endurance and long term cold endurance among others.
	In this work we investigate the feasibility of using infrared thermography as a QC technique for the fully loaded petals. Depending on the target surface, however, infrared measurements can be quite inaccurate and unreliable. Silicon, for example, is a tricky material to perform infrared measures on, and it is however the most important constituent material of the detector. That is why a good understanding of infrared thermography is crucial to achieve accurate temperature measurements.
	For simulating the sensor's thermal response at different operational parameters, thermal Finite Element Analyses (FEAs) are used. In particular, for the petal FEA models, exact design specifications of each component are considered. However, this simulation models have to be validated. For that purpose, numerous building institutes assembled and measured thermo-mechanical module prototypes.
	In this study, we use the thermo-mechanical petal prototype assembled at DESY aiming at two main objectives: to study the thermal performance of the petal design under	normal operation conditions and to perform infrared tests on the current petal thermo-mechanical prototype in order to provide the FEA team with a comparison point to validate their model. In order to do so, a reliable IR temperature determination is indispensable. To that end, a series of studies were first performed to gain better understanding of the Infrared Thermography technique:
		
	\begin{enumerate}
		\item Apparent reflected temperature determination.
		\item IR camera’s intrinsic spectral response scale factor determination.
		\item Viewing angle influence.
	\end{enumerate}
	
	Afterwards, a thermogram correction method was investigated in order to accurately measure the temperature of the Silicon surface of the sensors by accounting for the difference in emissivity among the petal constituents.
	This report contains the results of these studies and is structured as follows: In Chapter 1 a brief background on infrared thermography, needed for the subsequent sections, is given. In Chapter 2 a detailed description of the experimental setup used in this work is presented. In Chapter 3, the preliminary thermal studies carried out are detailed. Finally, in Chapter 4, the results of the thermogram calibration technique are presented and the comparison with the FEA simulations is discussed. 