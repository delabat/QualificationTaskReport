% % % Set the style for this file:
\pagestyle{introduction}

% % % Beginning of the section
\section*{\uppercase{Introduction}}\label{intro}
	\bigskip
	\bigskip
	Petals are the local support structures holding the ATLAS end-cap ITk strip sensors [1]. For the petal sensors, a series of tests are being designed to cover both Quality Assurance (QA) and Quality Control (QC). These include a series of tests on their assembled structure (through metrology surveys), their electrical functionality and their thermomechanical properties (including ASIC burn in, thermal cycling and long term cold tests). 
	As it's very important that, during the QC process, good care is taken not to damage the fragile components of the sensors, the thermal tests will be made by using infrared thermography, which is the most suitable technique to employ for this purposes. However, depending on the target surface, infrared measurements can be quite inaccurate and unreliable. Silicon, for example, is a tricky material to perform infrared measures on, and it's however the most important component on the detector sensors. That is why a good understanding of infrared thermography is crucial to achieve temperature measurements as accurate as possible.
	For simulating the sensor's response for different operational parameters, thermal Finite Element Analyses (FEAs) are used. In particular, for the petal FEA models, exact design specifications of each component are considered. However, this models have to be validated in order to be able to use the outputs. For that purpose, numerous building institutes assembled and measured thermo-mechanical modules prototypes. 
	In this study, we use the thermomechanical petal prototype assembled at DESY aiming at two main objectives: to contribute to the improvement of the QC and QA processes of the petal structures for the ATLAS end-caps and to perform infrared tests on the current petal thermomechanical prototype in order to provide the FEA team with a comparison point to validate their model. In order to do so, a reliable IR temperature determination is indispensable. To that end, a series of studies were performed to gain better understanding of the way in which those factors affect such measurements:
	
	\begin{enumerate}
		\item Emissivity correction/estimation methods for the silicon sensors.
		\item Apparent reflected temperature determination.
		\item Viewing angle influence.
		\item IR camera's intrinsic spectral response scale factor determination.
	\end{enumerate}
	
	This report contains the results of that studies and is structured as follows: In Chapter 1 a brief background on infrared thermography, needed for the subsequent sections is given. In Chapter 2 a detailed description of the experimental setup used in this work is presented. In Chapter 3, the analysis method used in the different studies is discussed. Finally, in Chapter 4, the results are presented accompanied by a small discussion. 