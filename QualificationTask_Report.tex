% % % Beginning of preamble...
\documentclass[10pt,a4paper]{report}

% % % packages to be used...
\usepackage[utf8]{inputenc}
\usepackage[english]{babel}
\usepackage{amsmath}
\usepackage{amsfonts}
\usepackage{amssymb}
\usepackage{graphicx}
\usepackage{lineno}
\usepackage{subfig}
\usepackage{caption}
\usepackage{enumerate}
\usepackage{graphicx}
\usepackage{float}
\usepackage{array}
\usepackage{fancyhdr}
\usepackage{datetime}
\usepackage{nameref}
\usepackage{ragged2e}
\usepackage{enumitem}
\usepackage[left=2.50cm, right=2.50cm, top=2.50cm, bottom=2.50cm]{geometry}
\usepackage{hyperref}

% % % hiperref options
\hypersetup{
    colorlinks=true,
    linkcolor=blue,
    filecolor=magenta,      
    urlcolor=cyan,
}

% % % Additional tex files to include:
% % % Define the syles....
\fancypagestyle{chapter-first-page}{
	\fancyhf{}
	\fancyhead[R]{\thepage}
	\fancyhead[C]{}
	\fancyhead[L]{}
	\fancyfoot[R]{}
	\fancyfoot[C]{[\draft \space \version]}
	\fancyfoot[L]{\usdate \today \space - \currenttime}
	\renewcommand{\headrulewidth}{0.4pt}
}
\fancypagestyle{introduction}{
	\fancyhf{}
	\fancyhead[R]{}
	\fancyhead[C]{}
	\fancyhead[L]{Introduction}
	\fancyfoot[R]{}
	\fancyfoot[C]{[\draft \space \version]}
	\fancyfoot[L]{\usdate \today \space - \currenttime}
	\renewcommand{\headrulewidth}{0.4pt}
}
\fancypagestyle{standard}{
	\fancyhf{}
	\fancyhead[R]{\thepage}
	\fancyhead[C]{}
	\fancyhead[L]{\nouppercase{\leftmark}}
	\fancyfoot[R]{}
	\fancyfoot[C]{[\draft \space \version]}
	\fancyfoot[L]{\usdate \today \space - \currenttime}
	\renewcommand{\headrulewidth}{0.4pt}
}
\fancypagestyle{conclusions}{
	\fancyhf{}
	\fancyhead[R]{}
	\fancyhead[C]{}
	\fancyhead[L]{Conclusions}
	\fancyfoot[R]{}
	\fancyfoot[C]{[\draft \space \version]}
	\fancyfoot[L]{\usdate \today \space - \currenttime}
	\renewcommand{\headrulewidth}{0.4pt}
}
\fancypagestyle{references}{
	\fancyhf{}
	\fancyhead[R]{}
	\fancyhead[C]{}
	\fancyhead[L]{References}
	\fancyfoot[R]{}
	\fancyfoot[C]{[\draft \space \version]}
	\fancyfoot[L]{\usdate \today \space - \currenttime}
	\renewcommand{\headrulewidth}{0.4pt}
}
\fancypagestyle{tableofcontents}{
	\fancyhf{}
	\fancyhead[R]{}
	\fancyhead[C]{}
	\fancyhead[L]{}
	\fancyfoot[R]{}
	\fancyfoot[C]{[\draft \space \version]}
	\fancyfoot[L]{\usdate \today \space - \currenttime}
	\renewcommand{\headrulewidth}{0pt}
}

% % % definitions...
\newcommand{\me}{Yasiel Delabat Díaz}
\newcommand{\supervisor}{Dr. Claire A. David}
\newcommand{\qttitle}{Infrared tests on the ATLAS thermomechanical petal prototype built at DESY.}
\newcommand{\qtabstract}{The infrared measurements on the thermomechanical petal prototype of the ATLAS end-cap strip detector were performed using a customized thermal chamber built at DESY. Using for the first time $CO_{2}$ cooling for the prototype's thermal cycles, temperatures of around -25\space$^\circ C$ were reached. After each cycle, it was observed that the sensors tend not to keep thermal memory (i.e. they are not damaged). 
Preliminary comparisons with FEA simulations also showed fairly similar behaviour with respect to the measurements performed on both sides of the petal. 
In addition, a thermographic correction squeme was investigated, aiming to use a mathematical approach for emissivity correction that would eliminate the necesity of covering the petal surface with high emissivity black tape. With that purpose, the IR camera's spectral response scale factor was estimated and the viewing angle influence in the measurements was studied founding it to be negligible.}
\newcommand{\draft}{DRAFT}
\newcommand{\version}{v0.3}

% % % Show line numbers (for DRAFT version)
\linenumbers

% % % New table columns alignment option:
\newcolumntype{P}[1]{>{\centering\arraybackslash}p{#1}}

% % % new command to repeat a character N times:
\makeatletter
\newcount\my@repeat@count
\newcommand{\myrepeat}[2]{%
  \begingroup
  \my@repeat@count=\z@
  \@whilenum\my@repeat@count<#1\do{#2\advance\my@repeat@count\@ne}%
  \endgroup
}
\makeatother

% % % Command to produce a line of custom length:
\newcommand*{\xdash}[1][3em]{\rule[0.5ex]{#1}{0.55pt}}

% % % Beginning of document...
\begin{document}

	% % % Add the cover page here...
	\begin{titlepage}
	
	% % % This is the header of the cover page
	\begin{center}
		\includegraphics[width=0.15\textwidth]{Figures/Logos/DESY-Logo-cyan-RGB_ger.jpg}
		\hfill
		\begin{minipage}[ct!]{0.6\linewidth}
			\centering
			\vspace{-2cm}
			\Large\textbf {QUALIFICATION TASK REPORT}\vspace{0.5cm}
			\Large {[\draft\space\version]}
		\end{minipage}
		\hfill
		\includegraphics[width=0.15\textwidth]{Figures/Logos/AT_atlaslogo_2015.pdf}
	\end{center}
	\vfill
	
	% % % Title
	\begin{center}
		\huge \textbf{\qttitle}
	\end{center}
	\vfill
	
	% % % Author
	\begin{center}
		\Large {By: \\ \me\space$^{a}$}
	\end{center}
	\vfill
	
	% % % Supervisor
	\begin{center}
		\Large {supervised by \\ \supervisor\space$^{a}$}
	\end{center}
	\vfill
		
	% % % Abstract
	\begin{center}
		\large \textbf{Abstract}
	\end{center}
	
	\large \qtabstract
	\vfill
		
	% Bottom of the page
	% % % Institution reference
	\hspace{-0.7cm} \xdash[12em] \\
	\textit{$^{a}$\space Deutsches Elektronen-Synchrotron (DESY)}
	
	% % % Date
	\begin{center}
		{\large \usdate \today }
	\end{center}
\end{titlepage}
	\clearpage	
	
	% % % Add the different parts here...
	% \tableofcontents
	% % % Set the style for this file:
\pagestyle{tableofcontents}

% % % Set the style for the first page:
\thispagestyle{tableofcontents}

% % % Table of Contents
\section*{{\Huge Contents}}\bigskip\bigskip

	\textbf{\nameref{intro}\dotfill\pageref{intro}}\bigskip
	
	\textbf{\ref{chapter1}\quad\nameref{chapter1}\dotfill\pageref{chapter1}}
	
		\qquad\ref{section1.1}\quad\nameref{section1.1}\dotfill\pageref{section1.1}
		
		\qquad\ref{section1.2}\quad\nameref{section1.2}\dotfill\pageref{section1.2}		

		\qquad\ref{section1.3}\quad\nameref{section1.3}\dotfill\pageref{section1.3}
		
		\qquad\ref{section1.4}\quad\nameref{section1.4}\dotfill\pageref{section1.4}\bigskip
	
	\textbf{\ref{chapter2}\quad\nameref{chapter2}\dotfill\pageref{chapter2}}
	
		\qquad\ref{section2.1}\quad\nameref{section2.1}\dotfill\pageref{section2.1}
		
		\qquad\ref{section2.2}\quad\nameref{section2.2}\dotfill\pageref{section2.2}
		
		\qquad\ref{section2.3}\quad\nameref{section2.3}\dotfill\pageref{section2.3}	
		
		\qquad\ref{section2.4}\quad\nameref{section2.4}\dotfill\pageref{section2.4}\bigskip
			
	\textbf{\ref{chapter3}\quad\nameref{chapter3}\dotfill\pageref{chapter3}}
	
		\qquad\ref{section3.1}\quad\nameref{section3.1}\dotfill\pageref{section3.1}	
		
		\qquad\ref{section3.2}\quad\nameref{section3.2}\dotfill\pageref{section3.2}
			
		\qquad\ref{section3.3}\quad\nameref{section3.3}\dotfill\pageref{section3.3}	
			
		\qquad\ref{section3.4}\quad\nameref{section3.4}\dotfill\pageref{section3.4}\bigskip
	
	\textbf{\ref{chapter4}\quad\nameref{chapter4}\dotfill\pageref{chapter4}}
	
		\qquad\ref{section4.1}\quad\nameref{section4.1}\dotfill\pageref{section4.1}	
		
		\qquad\ref{section4.2}\quad\nameref{section4.2}\dotfill\pageref{section4.2}	
		
		\qquad\ref{section4.3}\quad\nameref{section4.3}\dotfill\pageref{section4.3}	
		
		\qquad\ref{section4.4}\quad\nameref{section4.4}\dotfill\pageref{section4.4}
		
		\qquad\ref{section4.5}\quad\nameref{section4.5}\dotfill\pageref{section4.5}\bigskip
		
	\textbf{\nameref{concl}\dotfill\pageref{concl}}\bigskip
			
	\textbf{\nameref{referen}\dotfill\pageref{referen}}\bigskip
	\clearpage
	
	% % % Set the style for this file:
\pagestyle{introduction}

% % % Beginning of the section
\section*{\begin{flushright}{\uppercase{Introduction}}\end{flushright}}
	

	
	\clearpage
	
	% % % Set the style for this file:
\pagestyle{standard}

% % % Beginning of the chapter
\chapter{Notions of infrared thermography}\label{chapter1}

	% % % Set the style for the first page:
	\thispagestyle{chapter-first-page}
	
	\section{Infrared Thermography. Infrared radiation spectrum.}\label{section1.1}
	
		Infrared thermography (IRT) is a science dedicated to the acquisition and processing of thermal information from non-contact measurement devices. Infrared measuring devices acquire infrared radiation emitted by an object and transform it into an electronic signal. As a non-contact technique, IRT has some advantages in relation to other techniques for temperature measurement. For example, the temperature of extremely hot objects or dangerous products, such as acids, can be measured safely, keeping the user out of danger. It also provides protection for the object under investigation since there is no need to attach any temperature measuring device to it, which makes it much less invasive.
		
		Infrared (IR) radiation is the energy irradiated by a surface that has a temperature above the absolute zero \ref{ref2}. Within the electromagnetic spectrum, IR radiation is defined as the radiation band that spans from 0.75 $\mu$m to 1000 $\mu$m in wavelength (Figure \ref{fig1.1}). Much of the IR spectrum, however, is generally avoided for IRT applications due to atmospheric absorption. This absorption occurs mainly with H$_{2}$O and CO$_{2}$ molecules as they are well known for being good heat absorbers (Figure \ref{fig1.2}). According to this fact, the IR wavelength range is sometimes further divided into 5 additional categories:

		\begin{enumerate}[label={\Roman*.}]
			\item Near-infrared (NIR) from 0.75 $\mu$m to 1.4 $\mu$m.
			\item Short-wavelength infrared (SWIR) from 1.4 $\mu$m to 3 $\mu$m.
			\item Mid-wavelength infrared (MWIR) from 3 $\mu$m to 8 $\mu$m.
			\item Long-wavelength infrared (LWIR) from 8 $\mu$m to 15 $\mu$m.
			\item Far infrared (FIR) from 15 $\mu$m to 1000 $\mu$m.
		\end{enumerate}
		
		\begin{figure}[ht!]
			\centering
			\captionsetup{justification=centering,margin=2cm}
			\includegraphics[scale=0.75]{Figures/Chapter01/Spectrum-of-electromagnetic-radiation.png}
			\caption{Electromagnetic spectrum showing the portion corresponding to IR radiation.}\label{fig1.1}
		\end{figure}
		
		It is important to note that this classification is somewhat arbitrary and therefore can vary within literature. Most IR sensors are designed to work in the LWIR part of the spectrum since this is the range that minimizes these absorptions. In this study, as we use an IR camera sensitive only to the fourth type, we will refer to IR as the LWIR except otherwise explicitly stated.
				
		\begin{figure}[ht!]
			\centering
			\captionsetup{justification=centering,margin=2cm}
			\includegraphics[scale=0.35]{Figures/Chapter01/Transmission.jpg}
			\caption{An example of atmospheric transmission plot for infrared radiation. For some of the minima we can see the molecule responsible for the absorption.}\label{fig1.2}
		\end{figure}
		
		There are, of course, some caveats associated with the use of IRT. One of them is in fact related with its advantageous characteristic of being a non-contact technique: IR radiation has to travel some distance from the target surface to the infrared sensor passing through some media that can either let it pass almost completely or strongly attenuate it. Using vacuum would solve this particular issue but sometimes it is not possible to perform infrared measurements in vacuum. Additional complications are related with the intrinsic characteristics of the emitting material and the IR sensor's particular response to IR radiation. All these factors need to be "compensated" in order to obtain meaningful quantitative IR measurements. We will discuss the most important ones in the following sections.
		
	\section{Plank's law for blackbodies. IR radiation dissipation.}\label{section1.2}
	
		According to Planck’s law, the IR emissive power ($N$) of a blackbody\footnote{{\footnotesize A blackbody is an idealized physical body that absorbs all incident electromagnetic radiation, regardless of frequency or angle of incidence.}} at a temperature $T$, with a wavelength between $\lambda$ and $\lambda+d\lambda$ is given by Equation \ref{eq1.1}, where C$_{1}$ ($2hc^2$) and C$_{2}$ ($hc/k_{B}$) are constant, often called first and second radiation constants respectively \ref{ref3}.
		
		\begin{equation}\label{eq1.1}
			N_{b}(\lambda,T)d\lambda=\frac{C_{1} \cdot \lambda^{-5}}{\exp (\frac{C_{2}}{\lambda\cdot T}) -1} d\lambda
		\end{equation}\bigskip
		
		Here the subindex $b$ denotes the special case of the blackbody. Figure \ref{fig1.3} shows this functional dependence in terms of an equivalent magnitude (monochromatic\footnote{{\footnotesize Monochromatic here refers to “per wavelength interval”. Also referred to as “spectral”.}} irradiance) for six different temperatures \ref{ref4}. The gray line represents the displacement of the maximum for each temperature. Note that, as the temperature decreases, the maximum emissive power moves to higher wavelengths, this is known as the Wien’s displacement law \ref{ref5}.
		
		There are three ways by which the incoming emissive power may be dissipated: absorption, transmission and reflection \ref{ref6}. The fractions of the total radiant energy that are associated with each of these modes of dissipation are referred to as the \textit{absorptivity}, \textit{transmissivity} and \textit{reflectivity} of the body. Three parameters are used to describe these phenomena: the spectral absorptance $\alpha$, which is the fraction of the spectral emissive power absorbed by the object, the spectral reflectance $\rho$, which is the fraction of the spectral emissive power reflected by the object, and the spectral transmittance $\tau$, which is the fraction of the spectral emissive power transmitted by the object.
		
		\begin{figure}[ht!]
			\centering
			\captionsetup{justification=centering,margin=2cm}
			\includegraphics[scale=0.35]{Figures/Chapter01/PlankFunction.jpg}
			\caption{Planck’s law in terms of the  monochromatic irradiance of  a blackbody in thermal equilibrium at a given temperature. (a) Objects with a high temperature emit most of the radiation in the middle wave infrared; (b) Objects with a low temperature emit most of the radiation in the long wave infrared. The two parts of the graph are scaled differently on the y-axis.}\label{fig1.3}
		\end{figure}		
		
		These three parameters are, in general, wavelength dependent and their sum must be one at any given wavelength and surface temperature:
		
		\begin{equation}\label{eq1.2}
			\alpha + \rho + \tau = 1
		\end{equation}\bigskip
		
		If we regard the surface as \textit{opaque} to the IR radiation, it means that the transmission coefficient is $\tau \equiv 0$. For the rest of the report, and unless otherwise explicitly indicated, we will use the term "opaque" in reference to the surface property in the IR spectrum. Then we can rewrite Equation \ref{eq1.2} as:	
		
		\begin{equation}\label{eq1.3}
			\alpha + \rho = 1
		\end{equation}\bigskip
		
		In the following we will consider all surfaces as opaque for the derivation of the necessary formulae. To consider transmission (except in those cases where it is absolutely necessary) constitutes a limiting factor for the IRT accuracy since it is a difficult magnitude to estimate. Furthermore, the additional uncertainties that would be introduced make any quantitative IR analysis very difficult.
		\bigskip
		
	\section{Emissivity definition. Kirchhoff's law.}\label{section1.3}
		
		One of the most important concepts in IRT is \textit{emissivity} ($\varepsilon$). Emissivity of a surface at a temperature $T$ for a given wavelength $\lambda$ is defined as the ratio of the emissive power of a non-blackbody to the emissive power of a blackbody at the same temperature and both measured at the same wavelength:
		
		\begin{equation}\label{eq1.4}
			\varepsilon(\lambda,T)=\frac{N(\lambda,T)}{N_{b}(\lambda,T)}
		\end{equation}\bigskip
		
		Here $N(\lambda,T)$ is the emissive power of the non-black body object. Being an idealized representation, the blackbody will have the maximum possible emissivity value of 1 (perfect emitter), while any other real surface will have emissivity values in the range $0 < \varepsilon < 1$. Thus, Equation \ref{eq1.4} tells us that real body emits only a fraction of the thermal energy that would be emitted by a blackbody at the same temperature (Figure \ref{fig1.4}). If the emissivity is constant and independent of the wavelength, the body is called a \textit{graybody}.
		
		\begin{figure}[ht!]
			\centering
			\captionsetup{justification=centering,margin=2cm}
			\includegraphics[scale=0.177]{Figures/Chapter01/BlackAndGreybodyComparison02.jpg}
			\includegraphics[scale=0.28]{Figures/Chapter01/BlackAndGreybodyComparison01.jpg}
			\caption{Relationship between the blackbody spectral radiation, the radiation emitted by a graybody and a selective emitter at the same temperature. (a) By definition, emissivity of the blackbody is 1 while other non-idealized surfaces have emissivities between 0 and 1. (b) It can be seen how the blackbody yields the maximum radiance at any given wavelength.}\label{fig1.4}
		\end{figure}
		
		\bigskip
		The emissivity of real objects, however, varies with respect to wavelength and therefore they cannot be considered graybodies. In fact, it might also depend on many other factors such as temperature and viewing angle, as we will discuss later. However, it is usually assumed that for short wavelength intervals, the emissivity can be considered constant. This assumption is used to treat real objects as graybodies in order to avoid the associated mathematical complications of emissivity estimation. For this reason, surface emissivities are often computed as the average of the emissivity through the wavelength interval in which the infrared sensor works. This average is also possible because the emissivity is a slow-varying function of wavelength for solid objects. However, this does not apply to other cases, such as gases or liquids.
		
		A good illustration of the effect of emissivity in the measurement of the real temperature of a surface is presented in Figure \ref{fig1.5}. This corresponds to the so-called Leslie’s Cube. Even though this cube is filled with hot water and all the faces are at the same temperature we can see a huge difference in the temperature measured by the IR sensor. This is due to the difference in emissivity of the two front faces. One is painted with a high emissivity black paint (could be any color) and therefore the apparent temperature is closer to the real temperature of the water, while the other face has low emissivity (very reflective) and the apparent temperature is almost entirely due to the reflected component (see the heat from the hand reflected in the surface).
		
		\begin{figure}[ht!]
			\centering
			\captionsetup{justification=centering,margin=2cm}
			\includegraphics[scale=0.55]{Figures/Chapter01/LesliesCube.jpg}
			\includegraphics[scale=0.52]{Figures/Chapter01/LesliesCube2.jpg}
			\caption{Infrared (Left) and visible (Right) images of a Leslie’s cube. The cube is filled with hot water and all the faces are at the same temperature. However,  one of the faces is coated with a high emissivity paint (black) and the other has been polished (low emissivity).}\label{fig1.5}
		\end{figure}
		
		If a blackbody is surrounded by an isothermal black enclosure of the same temperature, then, in thermodynamic equilibrium, such blackbody will absorb 100\% of the radiation emitted by the enclosure ($\alpha=1$). At the same time it will emit 100\% of its own thermal radiation since it has $\varepsilon=1$. Under those circumstances, the following relationship holds:
		
		\begin{equation}\label{eq1.5}
			\alpha \equiv \varepsilon
		\end{equation}\bigskip	
		
		This is known as the Kirchhoff’s law of thermal radiation. This law, derived here for the special case of the blackbody, can be extended as well to non-blackbodies and basically tells us that the emissivity and absorptivity of any material are equal at any specified temperature and wavelength. Thus, we can rewrite Equation \ref{eq1.2} as:
		
		\begin{equation}\label{eq1.6}
			\varepsilon + \rho + \tau = 1
		\end{equation}\bigskip	
	
		And for the case in which transmissivity $\tau=0$, this becomes:
		
		\begin{equation}\label{eq1.7}
			\varepsilon = 1 - \rho
		\end{equation}\bigskip	
	
	\section{IR measurements calibration.}\label{section1.4}
	
		Usually, when we take an IR picture of an object, at room temperature for instance, we intuitively expect the entire image to be of only one color (corresponding to the ambient temperature). However, this is rarely the case. It is important to point out the fact that the IR sensor/camera does not gives us the “real” temperature values but rather an “apparent” one. We refer to these temperatures as \textit{apparent temperatures} since it is necessary later on to correct them due to the presence of several factors that provoque deviations in what the IR sensor perceives. The most important ones can be classified as follows:

		\begin{enumerate}[label={\arabic*.}]
			\item \textbf{Intrinsic object properties}: emissivity, reflectivity, transmissivity and polishing of the surface.
			\item \textbf{IR sensor properties}: Spectral Response Function of the IR camera in the wavelength range in which it operates.
			\item \textbf{Other factors}: ambient conditions (temperature, RH), apparent reflected temperature and viewing angle of the IR camera with respect to the object.
		\end{enumerate}
		
		To understand how all these factors come into play we can consider the situation depicted in Figure \ref{fig1.6}.  Let's assume that we have a target that is placed in front of a generic IR sensor. We are interested in measuring the amount of radiation coming from the target due to emission alone, since this corresponds to the only measure of the “real” temperature of our object. Using Equation \ref{eq1.4} we can estimate the amount of emitted IR radiation ($N$) as: 
		
		\begin{equation}\label{eq1.8}
			N(\lambda,T)=\varepsilon(\lambda,T) \cdot N_{b}(\lambda, T)
		\end{equation}\bigskip
		
		Here $T$ is the actual (real) temperature of the target.
			
		\begin{figure}[ht!]
			\centering
			\captionsetup{justification=centering,margin=2cm}
			\includegraphics[scale=0.38]{Figures/Chapter01/SchematicsOfIRRadiation.jpg}
			\caption{Schematics of the IR radiation interaction with a target surface: (A) radiation from an external source being partially absorbed by the target, (B) part of the radiation from A that is is reflected in the direction of the IR camera, (C) thermal  radiation emitted from the target and (D) external source of radiation transmitted through the target.}\label{fig1.6}
		\end{figure}
		
		In addition, we will have two more contributions to the amount of IR radiation flying in the direction of the sensor. These other contributions are produced by additional sources of heat. The first is the amount of radiation coming from external heat sources that is reflected on the target surface. This contribution is given by:
		
		\begin{equation}\label{eq1.9}
			N_{ref}(\lambda,T)=\rho(\lambda,T) \cdot N_{b}(\lambda, T_{r})
		\end{equation}\bigskip
		
		Where $T_{r}$ is the \textit{apparent reflected temperature} of the target surface. This is an irreducible source of IR background, even with an isolated thermal chamber, and is therefore always present in the estimation of the real surface temperature. In this sense, we can regard the interior walls of the isolation chamber (and all other equipment inside) as “external” heat sources and, as the temperature of such objects should be close to the room temperature, this apparent reflected temperature is often taken as the ambient temperature. We will discuss further on this temperature and its estimation method in Section \ref{section3.1}. The second source of additional IR radiation is that part of the radiation coming from a heat source behind the target that is transmitted through the target itself. This contribution is determined as:
		
		\begin{equation}\label{eq1.10}
			N_{trans}(\lambda,T)=\tau(\lambda,T) \cdot N_{b}(\lambda, T_{t})
		\end{equation}\bigskip	
		
		where $T_{t}$ is the \textit{apparent transmitted temperature}. This source of IR background can be avoided by placing the target in a thermal enclosure (chamber), leaving all external heat sources outside. However, for complex objects composed by layers of different materials, this factor can appear if, for example, the surface material (facing the IR sensor) is somewhat transmissive and the heat from other layers reaches them and passes through.
		
		The combination of all these three factors is then the total amount of IR radiation coming from the target that heads towards the camera:
		
		\begin{equation}\label{eq1.11}
			N_{total}(\lambda,T)=N(\lambda,T)+N_{ref}(\lambda,T_{r})+N_{trans}(\lambda,T_{t})
		\end{equation}\bigskip
		
		Neglecting the transmission component (assuming the object opaque), Equation \ref{eq1.11} becomes:
		
		\begin{equation}\label{eq1.11a}
		N_{total}(\lambda,T)=N(\lambda,T)+N_{ref}(\lambda,T_{r})=\varepsilon(\lambda,T) \cdot N_{b}(\lambda, T)+[1-\varepsilon(\lambda,T)] \cdot N_{b}(\lambda, T)
		\end{equation}\bigskip
		
		where we have used Equation \ref{eq1.7} to express everything in terms of emissivity. If we also consider that this $N_{total}(\lambda,T)$ is not attenuated in the air path to the IR sensor\footnote{{\footnotesize We can see from Figure 1.2 (for 10 m) that the air transmissivity is nearly 1 in the LWIR region of the IR spectrum. Additionally, as we will see later on, our experimental setup was maintained at low humidity to further improve the transmissivity of the environment.}} then the emissive power at temperature $T$ registered by the sensor for a given wavelength is:
		
		\begin{equation}\label{eq1.12}
			N_{meas}(\lambda,T)=R(\lambda) \cdot N_{total}(\lambda,T)
		\end{equation}\bigskip
		
		where $R(\lambda)$ is a correction scale factor introduced to account for the fact that our IR camera is sensitive only in the spectral range from 7.5 $\mu$m to 14 $\mu$m, and even in that range it is also not 100\% sensitive to the incoming radiation. It is often called sensor’s \textit{Spectral Response Function} or sensor’s \textit{Filter Function} \ref{ref7}. Note that this factor only depends on the IR sensor’s “efficiency” for a given wavelength and it can be estimated as the ratio of the power registered by the sensor and the power calculated using Equation \ref{eq1.12}.
		
		To obtain then the total (integrated over all wavelengths) signal processed by the IR sensor we have to integrate Equation \ref{eq1.12} over the wavelength range that our sensor is sensitive to (from $\lambda_{1}$=7.5 $\mu$m to $\lambda_{2}$=14 $\mu$m):
		
		\begin{equation}\label{eq1.13}
			N_{meas}(T)= \int_{\lambda_{1}}^{\lambda_{2}} N_{meas}(\lambda,T) d\lambda = \int_{\lambda_{1}}^{\lambda_{2}} R(\lambda) \cdot N_{total}(\lambda,T) d\lambda
		\end{equation}		
		
		\begin{equation}\label{eq1.14}
			N_{meas}(T)= \int_{\lambda_{1}}^{\lambda_{2}} R(\lambda) \cdot \varepsilon(\lambda,T) \cdot N_{b}(\lambda,T) d\lambda + \int_{\lambda_{1}}^{\lambda_{2}} R(\lambda) \cdot [1- \varepsilon(\lambda,T)] \cdot N_{b}(\lambda,T_{r}) d\lambda
		\end{equation}\bigskip	
		
		As we don’t know the functional form of $R(\lambda)$ or $\varepsilon(\lambda,T)$ with respect to $\lambda$ we can come around this by applying the integral mean value theorem and use averaged value instead. Since both $R(\lambda)$ and $\varepsilon(\lambda,T)$ should be slow-varying functions of the wavelength, there must be certain value $\xi \in [\lambda_{1},\lambda_{2}]$ such that we can then express Equation \ref{eq1.14} as:
		
		\begin{equation}\label{eq1.15}
			N_{meas}(T)= R(\xi) \cdot \varepsilon(\xi,T) \cdot \int_{\lambda_{1}}^{\lambda_{2}} N_{b}(\lambda,T) d\lambda + R(\xi) \cdot [1- \varepsilon(\xi,T)] \cdot \int_{\lambda_{1}}^{\lambda_{2}} N_{b}(\lambda,T_{r}) d\lambda
		\end{equation}	
		
		\begin{equation}\label{eq1.16}
			N_{meas}(T)= R \cdot \bar{\varepsilon} \cdot I_{1}(T) + R \cdot [1- \bar{\varepsilon}] \cdot I_{2}(T_{r})
		\end{equation}\bigskip
		
		where $\bar{\varepsilon}=\varepsilon(\xi,T)$ is the averaged emissivity in the wavelength region in which the IR camera is most sensitive and $I_{1}(T)$ and $I_{2}(T)$ can be calculated numerically:
		
		\begin{equation}\label{eq3.10}
			I_{1}(T)=\int_{\lambda_{1}}^{\lambda_{2}} N_{b}(\lambda,T) d\lambda
		\end{equation}
		
		\begin{equation}\label{eq3.11}
			I_{2}(T)=\int_{\lambda_{1}}^{\lambda_{2}} N_{b}(\lambda,T_{r}) d\lambda
		\end{equation}\bigskip
		
		$R(\xi)$ can also be treated as a constant scale factor ($R$), which only depends on the IR sensor (See results in Section \ref{section4.4}).
		
		Even though in some cases we can assume emissivity as a constant that only depends on the specific material of the surface under investigation, in general, it also depends on the viewing angle of the camera with respect to the normal of the surface. 
		Blackbodies behave like perfect isotropically diffuse emitters, that is, for any surface emitting radiation, the \textit{radiance}\footnote{{\footnotesize The term \textit{radiance} refers to the amount of IR radiation emitted, transmitted or reflected (although in this work we focus only in the emitted component) by a surface per unit of area and solid angle. Therefore it is a \textit{directional} quantity. Should not be confused with \textit{irradiance}, which is the corresponding non-directional magnitude.}} of the emitted radiation is independent of the direction into which it is emitted \ref{ref8}. However, real surfaces, in addition to the lower emission rate (given by lower emissivity) also show angular dependence in the radiance (Figure \ref{fig1.7} right).
		
		\begin{figure}[ht!]
			\centering
			\captionsetup{justification=centering,margin=2cm}
			\includegraphics[scale=0.35]{Figures/Chapter01/AngularDistributionSchematics02.jpg}
			\includegraphics[scale=0.35]{Figures/Chapter01/AngularDistributionSchematics01.jpg}
			\caption{Schematics of the angular dependence of the emissive power of a surface in relation to the blackbody.}\label{fig1.7}
		\end{figure}
		
		Consequently, the emissive power also depends on the viewing angle between the IR camera and the surface's normal due to the definition of the surface's unit of area. By construction, this area is the projection of the surface's unit of area into the plane perpendicular to the direction of emission (Figure \ref{fig1.7} left). As a result, for an angle of 90$^\circ$ with respect to the normal of the surface the emissive power will be minimal while for a viewing angle of 0$^\circ$ will be maximal. As shown in the figure, this is purely a geometrical factor.
		Depending on the surface, the emissive power of the emitted radiation can be approximated with a cosine law with good accuracy. However, in general, this is not the case, and in most cases a study must be carried out to estimate the angular dependence of the emissivity.
		
	\clearpage
	
	% % % Set the style for this file:
\pagestyle{standard}

% % % Beginning of the chapter
\chapter{Experimental setup}\label{chapter2}

	% % % Set the style for the first page:
	\thispagestyle{chapter-first-page}
	
	\section{Thermo-mechanical petal}\label{section2.1}
	
		Both ATLAS ITk strip end-caps (Figure \ref{fig2.1} left) are composed by six disks, which individually can hold 32 local support structures called petals (Figure \ref{fig2.1} right). The petal structure is the frame where the end-cap sensors are mounted. It provides the mechanical support needed by the sensors as well as the electronics and cooling services. Each petal holds 9 sensors (Figure \ref{fig2.1} right). 
		In this study we use a thermo-mechanical prototype equipped with dummy electronics modules, blank silicon wafers and mechanical components to be a thermal equivalent to the final structure. However, this prototype contains materials that can not be considered as IR friendly, mainly because of their low emissivity. An example of this is in fact the Silicon surface, which is also the object of most interest for this study. Consequently, great effort has been made to compensate or "correct" for this fact (See Chapter \ref{chapter4}). In that sense, it is important to mention that the Silicon wafers used in the thermo-mechanical petal prototype built at DESY have different IR properties on each side. On one side, the Silicon surface of the wafer looks shiny and polished, while the other looks more unpolished. This is due to the wafer processing required to manufacture the sensors.
		On one side of the petal, the Silicon sensors were placed with the polished side visible. On the other side of the petal, the unpolished side of the Silicon sensors is the visible one. We call these the polished and unpolished sides of the petal respectively.
				
		\begin{figure}[ht!]
			\centering
			\captionsetup{justification=centering,margin=2cm}
			\includegraphics[scale=0.25]{Figures/Chapter02/EndCap.jpg}
			\includegraphics[scale=0.26]{Figures/Chapter02/PetalDesign.jpg}
			\caption{ATLAS strip detector end-cap (left) showing petal structures and a schematic view of the petal showing the 6 different modules (right).}\label{fig2.1}
		\end{figure}
		
		The nine modules of each petal side are glued on the petal core (carbon facesheets) in six subsegments referred to as rings (R0 - R5). Each module contains the following components:
		
		\begin{itemize}
			\renewcommand{\labelitemi}{$\diamond$}
			\item Blank Si, laser cut (320 $\mu$m thick).
			\item FR4 PCBs (200 $\mu$m thick).
			\item Glass ASICs with heater pattern and bonding pads, glued with UV glue, wire-bonded to bare silicon.
			\item Real DC-DC converters, based on commercial LTC360 ASIC on a custom board.
			\item Potentiometer to adjust power input/output.
			\begin{itemize}
			\renewcommand{\labelitemi}{$\bullet$}
				\item Powered through bus tape power lines.
			\end{itemize}
			\item In the case of R0 module: v0 assembly tools, real hybrids, glass ASICs.
		\end{itemize}
		
		In addition, the prototype core was built using real materials from to preliminary petal designs (bus tape, honeycomb, Ti cooling pipe) and a dummy EoS (not present initially but recently installed) per side. Some of the above-mentioned components can be seen in the Figure \ref{fig2.2}.
		
		\begin{figure}[ht!]
			\centering
			\captionsetup{justification=centering,margin=2cm}
			\includegraphics[scale=0.35]{Figures/Chapter02/PetalConstruction.jpg}
			\caption{Thermo-mechanical prototype of the petal  used in the IR measurements.}\label{fig2.2}
		\end{figure}\bigskip
		
	\section{Custom Thermal Chamber. }\label{section2.2}
	
		For the infrared measurements, the thermomechanical petal prototype was placed inside a customised thermal chamber where, on one end, the petal is vertically inserted in a custom rail system (Figure \ref{fig2.3} left) and on the opposite end the IR camera is mounted on a mobile platform that can move horizontally and vertically thanks to an Arduino controlled Gantry System (Figure \ref{fig2.3} right). Temperature and relative humidity (RH) inside the chamber are monitored using three SHT21 sensors connected  to a Raspberry Pi. 
		
		\begin{figure}[ht!]
			\centering
			\captionsetup{justification=centering,margin=2cm}
			\includegraphics[scale=0.25]{Figures/Chapter02/ChamberBack.jpg}
			\includegraphics[scale=0.26]{Figures/Chapter02/CamberFront.jpg}
			\caption{Image of both ends of the thermal chamber showing the petal already in place and the IR camera also in position.}\label{fig2.3}
		\end{figure}
		
		The use of the chamber has two main advantages: first, it provides shielding for the object under investigation against external heat sources such as ceiling lamps, computers and other electronic devices used in the setup; second, it is used as an enclosure where we can flush nitrogen in order to reduce the moisture and in doing so preventing condensation from ambient air on the cooled sensors, which would irreversibly damage the prototype. Additionally, as the N$_{2}$ fills the interior of the chamber, the main sources of IR absorption are displaced: the H$_{2}$O and CO$_{2}$ molecules (See Chapter \ref{chapter1}). This results in an even better IR absorption reduction in the path between the surface of the petal and the IR camera.
		In order to perform the IR measurements avoiding the heat of IR camera itself to be reflected back by the petal surface\footnote{{\footnotesize This is known as the Narcissus effect and it is an important source of background for the IR measurements if not properly handled.}}, the IR camera is positioned in the chamber in such a way that it faces the petal with an angle of incidence between 40 and 45 degrees (Figure \ref{fig2.4}).	
		
		\begin{figure}[ht!]
			\centering
			\captionsetup{justification=centering,margin=2cm}
			\includegraphics[scale=0.25]{Figures/Chapter02/NarcissusEffect.jpg}
			\caption{The IR camera is placed at some angle with respect to the plane of the petal to avoid the Narcissus effect.}\label{fig2.4}
		\end{figure}\bigskip
		
	\section{IR camera.}\label{section2.3}
	
		For this study a VarioCAM High Resolution (hr) IR camera from InfraTec GmbH was used (Figure \ref{fig2.5}). The VarioCAM$\textregistered$ hr is a thermographic system for the long wave infrared spectral range of 7.5 $\mu$m to 14 $\mu$m (LWIR). The lens images the object scene onto a microbolometer array at a resolution of 640 $\times$ 480 pixels. The electrical signal of the detector arrays is further processed by the internal electronics, which consists in converting the modified pixel resistance due to the incoming radiation into temperature. The electronics also contains all the functions necessary for camera operation, such as activation of the microbolometer array, A/D conversion, offset and gain correction, defective pixel treatment, video and PC interfaces \ref{ref9}.
		
		\begin{table}[ht!]
    		\begin{minipage}[b]{0.4\linewidth}
  				\centering
  				\captionsetup{justification=centering, margin=0.5cm}
  				\includegraphics[scale=0.3]{Figures/Chapter02/PictureOfIRCamera.jpg}
  				\captionof{figure}{VarioCAM\textregistered\space hr (640 x 480 pixels) IR camera  from InfraTec \textcopyright\space GmbH.}\label{fig2.5}
    		\end{minipage}
    		\begin{minipage}[b]{0.7\linewidth}
    			\centering
  				\captionsetup{justification=raggedright}
        		\caption{VArioCam hr technical data.}\label{tab2.1}
				\begin{tabular}{p{0.35\linewidth}p{0.35\linewidth}}\hline
					\textbf{Temperature measuring range} & (-40 ... 1.200) $^{\circ}C$, optional $>$ 2.000 $^{\circ}C$ \\ \hline 
					\textbf{Temperature resolution @ 30 $^{\circ}C$} &  better than 0.08 K, up to 0.05 K (premium mode) \\ \hline
					\textbf{Emissivity} & Adjustable from 0.1 to 1.0, in increments of 0.01 \\ \hline
					\textbf{Detector} &  uncooled microbolometer Focal Plane Array \\ \hline
					\textbf{A/D conversion} &  16 bit \\ \hline 
					\textbf{Operation temperature} & (-15 ... 50) $^{\circ}C$ \\ \hline
					\textbf{Humidity during operation and storage} & 5\% to 95\%, non-condensing \\ \hline
					\textbf{Shock resistance} & 25 G, IEC 68-2-29 \\ \hline
				\end{tabular}
    		\end{minipage}
		\end{table}
		
		The camera’s accuracy for temperature measurement (reported by manufacturer) is $\pm 1.5 K$ in the range from 0$^\circ$C to 100$^\circ$C and $\pm2$\% anywhere outside that range. Some additional technical data is presented in Table \ref{tab2.1}.
		
		The data acquisition is performed using the VarioCam hr control software IRBIS$\textregistered$ professional v3.1 (Figure \ref{fig2.6}). Each image is computed as an average of 100 acquisitions regularly taken during five seconds in order to reduce uncertainty due to pixels noise. All thermograms are recorded in units of absolute temperature (K) and emissive power ($W/m^2$) for further offline analysis.
		
		\begin{figure}[ht!]
			\centering
			\captionsetup{justification=centering,margin=2cm}
			\includegraphics[scale=0.25]{Figures/Chapter02/IRBISimage.jpg}
			\caption{Screenshot of the camera’s control software IRBIS$\textregistered$ 3.1.}\label{fig2.6}
		\end{figure}\bigskip
		
	\section{Setup configuration.}\label{section2.4}
	
		In this study, two thermal tests performed using the setup described below are considered, corresponding to both sides of the petal and measured with the latest experimental configuration. 
		In the following, this will be referred to as “front side test” (unpolished side test) and “back side test” (polished side test) in allusion to the specific side of the petal tested. 
		
		Much of the improvements of the experimental setup came from experiences of preliminary tests when we realized that, for example, the heat from the Gantry system was being reflected on the petal’s surface and registered by the IR camera. Thus, a curtain from an IR opaque black fabric was placed between the petal and the Gantry system (leaving a small hole for the camera lens) to suppress this effect (Figure \ref{fig2.7} right). In addition, other improvements to the chamber’s insulation were made to better control the ambient conditions inside at lower temperatures (Figure \ref{fig2.7} left). 
	
		\begin{figure}[ht!]
			\centering
			\captionsetup{justification=centering,margin=2cm}
			\includegraphics[scale=0.185]{Figures/Chapter02/ExperimentalSetup.jpg}
			\includegraphics[scale=0.22]{Figures/Chapter02/BlackCurtine.jpg}
			\caption{Experimental setup used for the petal’s thermal tests. Right: a view of the chamber with new insulation, laptop for data acquisition, power supplies and Keithley (next to the laptop) and TRACI (Red box). Left: black curtain installation.}\label{fig2.7}
		\end{figure}
	
		This is particularly important for an accurate estimation of the apparent reflected temperature (See Section \ref{section3.1}). In addition, the DC-DC converters in modules R2 and R3 were covered with 3D-printed caps and black tape due to the fact that their heat created a “halo” of hot air (Figure \ref{fig2.8} right) around them that blurred the thermograms of that area of the petal (Figure \ref{fig2.8} left).
		
		For cooling the petal the \textit{Transportable Refrigeration Apparatus for CO$_{2}$ Investigation} (TRACI) Version 2 (100W) with Lewa pump was used (Figure \ref{fig2.7} left). This was the first time that a themo-mechanical petal prototype was cooled down using by-phase CO$_{2}$ in our setup. Previously, a water-glycol chiller was used, which greatly limited the lowest temperature that we were able to reach.
	
		\begin{figure}[ht!]
			\centering
			\captionsetup{justification=centering,margin=2cm}
			\includegraphics[scale=0.33]{Figures/Chapter02/DCDC_covers.jpg}
			\includegraphics[scale=0.46]{Figures/Chapter02/HaloThermogram.jpg}
			\caption{Unpolished side of the petal showing the DC-DC covers (left) and the thermogram where that “halo” of hot air around them is visible inside the circle (right).}\label{fig2.8}
		\end{figure}
	
		By controlling the CO$_{2}$ pressure in the experiment we were able to adjust the desired temperature working point. Using this method temperatures of near -25$^{\circ}$C were reached.
		
		The petal is equipped with nine dummy circuits simulating the heat emitted by the readout electronics (6 module electronics + 1 EoS) per side. An important aspect for the prototype operation is that the powerboards should have a constant power consumption of $\sim$25W and the EoS $\sim$3W on each side of the petal. To power the powerboards and EoS of both sides, two TTICPX400 power supply units were used. The modules voltage is set to 10.5V and the current to 2.5A. In the case of the EoS the current is set to 1.0A (for both sides) and the voltage to 3V. However, as the resistance of the circuit changes with temperature we had to vary the voltage accordingly to keep the 3W of power consumption constant. For the front side test we did it manually but for the back side test, as part of the Summer Student program, the student involved in the IR project was able to automatize the process by creating a program that automatically varied the voltage input to keep a steady 3W power consumption \ref{ref10}.
		
		In addition, a Keithley 2700 multimeter was used to register the readings from additional PT100 thermocouples using 4 wire sensing and some other important TRACI parameters like, for example, CO$_{2}$ flow, CO$_{2}$ temperature before experiment (petal), CO$_{2}$ temperature after experiment, pressure set-point and pressure of the CO$_{2}$ in the experiment.
		
		The additional thermocouples were placed as follows (for the front side test): 2 on the inlet/outlet pipes (glued), 2 in R3 module silicon surface (maintained by a piece of high emissivity black tape): 1 between the ASICs and 1 in the corner next to R4 (Figure \ref{fig2.9} top). For the four extra thermocouples were placed in R0, R1, R4 and R5 as shown in Figure \ref{fig2.9} (bottom).
		
		\begin{figure}[H]
			\centering
			\captionsetup{justification=centering,margin=2cm}
			\includegraphics[scale=0.47]{Figures/Chapter02/R3PT100CloseUp.jpg}
			\includegraphics[scale=0.45]{Figures/Chapter02/AdditionalPT100perModule.jpg}
			\caption{Close-up of the unpolished side of the petal showing the 2 PT100 thermocouples attached to R3 module (top). Polished side of the petal ready for the thermal test. The additional PT100 thermocouples placed at each module (except R2) are visible (bottom).}\label{fig2.9}
		\end{figure}

	\clearpage
	
	% % % Set the style for this file:
\pagestyle{standard}

% % % Beginning of the chapter
\chapter{Infrared thermograms analysis}\label{chapter3}

	% % % Set the style for the first page:
	\thispagestyle{chapter-first-page}

	\section{Factors affecting the IR measurements.}\label{section3.1}
	
		When we take an IR picture of any object at room temperature, for instance, we intuitively expect the entire image to be of only one color (corresponding to the ambient temperature). However, this is rarely the case. There  are several factors that can make the IR temperature readings to move away from the real values, the most important are:
		
		\begin{enumerate}[label={\arabic*)}]
			\item Different emissivity values of the objects in the image.
			\item Reflectivity and Transmissivity of the object.
			\item Spectral Response Function of the IR camera in the wavelength range in which it operates.
			\item Ambient conditions. Apparent reflected temperature.
			\item Viewing angle of the IR camera with respect to the object.
		\end{enumerate}
		
		To understand how all this factors come into play we can imagine the following situation (Figure \ref{fig3.1}).  Let's assume that we have a target which is placed in front of a generic IR sensor. We are interested in measure the amount of radiation coming from the target due to emission alone, since this will be the only measure of the “real” temperature of our object. Form Equations \ref{eq1.1} and \ref{eq1.4} we can estimate the amount of emitted IR radiation as: 
		
		\begin{equation}\label{eq3.1}
			\varepsilon(\lambda,T) \cdot N_{b}(\lambda,T)
		\end{equation}	
		
		Here $T$ is the actual (real) temperature of the target. In addition, we will have two more contributions to the amount of IR radiation flying in the direction of the sensor. This other contributions are produced by additional sources of heat. 
		The first is the amount of radiation coming from external heat sources that is reflected in the target surface. This contribution is given by:
		
		\begin{equation}\label{eq3.2}
			\rho(\lambda,T) \cdot N_{b}(\lambda,T_{r})
		\end{equation}	
	
		Where $T_{r}$ is the \textit{apparent reflected temperature} of the target surface. This is an irreducible source of IR background, even with an isolated thermal chamber, and is therefore always present in the estimation of the real surface temperature. In this sense, we can regard the interior walls of the isolation chamber (and all other equipment inside) as "external" heat sources and, as the temperature of such objects should be close to the room temperature, this apparent reflected temperature is often taken as the ambient temperature. We will discuss further on this temperature and its estimation method in the next paragraph.
		The second source of additional IR radiation is that part of the radiation coming from a heat source behind the target that is transmitted through the target itself. This contribution is determined as:
		
		\begin{equation}\label{eq3.3}
			\tau(\lambda,T) \cdot N_{b}(\lambda,T_{t})
		\end{equation}	
		
		Where $T_{t}$ is the \textit{apparent transmitted temperature}. This source of IR background can be avoided by place the target in a thermal enclosure (chamber), leaving all external heat sources outside. However, for complex objects (as the Petal) composed by layers of different materials, this factor can appear if, for example, the surface material (facing the IR sensor) is somewhat transmissive and the heat from other layers reaches them and passes through.
		
		\begin{figure}[ht!]
			\centering
			\captionsetup{justification=centering,margin=2cm}
			\includegraphics[scale=0.30]{Figures/Chapter03/SchematicsOfIRRadiation.pdf}
			\caption{Schematics of the IR radiation interaction with a target surface: (A) radiation from an external source being partially absorbed by the target, (B) part of the radiation from A which is reflected in the direction of the IR camera, (C) thermal  radiation emitted from the target and (D) external source radiation transmitted through the target.}\label{fig3.1}
		\end{figure}
		
		The combination of all three of this factors is then the total amount of IR radiation emanating from the target surface per unit of time and area. Neglecting the transmission component, this amount is:
		
		\begin{equation}\label{eq3.4}
			N_{total}(\lambda,T)=\varepsilon(\lambda,T) \cdot N_{b}(\lambda,T)+ [1- \varepsilon(\lambda,T)] \cdot N_{b}(\lambda,T_{r})
		\end{equation}	
		
		Where we have used Equation \ref{eq1.7} to express everything in terms of emissivity. If we also consider that this $N_{total}(\lambda,T)$ is not attenuated in the air path to the IR sensor then the emissive power registered by the sensor is:
		
		\begin{equation}\label{eq3.5}
			N_{meas}(\lambda,T)= R(\lambda) \cdot N_{total}(\lambda,T)
		\end{equation}	
		
		Where $R(\lambda)$ is the sensor's \textit{Spectral Response Function}, also known as Filter Function, accounting for the fact that the sensor is intrinsically not 100\% efficient in processing the incoming spectral power nor in the entire wavelength spectrum. As mentioned in Chapter 2, our IR camera is sensitive only in the spectral range from 7.5 $\mu$m to 14 $\mu$m, but even in that range it's also not 100\% sensitive to the incoming radiation. 
		To obtain then the total (for all wavelengths) signal processed by the IR sensor we have to integrate Equation \ref{eq3.5} over the wavelength range that our sensor is sensitive to (from $\lambda_{1}$=7.5 $\mu$m to $\lambda_{1}$=14 $\mu$m):
		
		\begin{equation}\label{eq3.6}
			N_{meas}(T)= \int_{\lambda_{1}}^{\lambda_{2}} N_{meas}(\lambda,T) d\lambda = \int_{\lambda_{1}}^{\lambda_{2}} R(\lambda) \cdot N_{total}(\lambda,T) d\lambda
		\end{equation}		
		
		\begin{equation}\label{eq3.7}
			N_{meas}(T)= \int_{\lambda_{1}}^{\lambda_{2}} R(\lambda) \cdot \varepsilon(\lambda,T) \cdot N_{b}(\lambda,T) d\lambda + \int_{\lambda_{1}}^{\lambda_{2}} R(\lambda) \cdot [1- \varepsilon(\lambda,T)] \cdot N_{b}(\lambda,T_{r}) d\lambda
		\end{equation}	
		
		As we don't know the functional form of $R(\lambda)$ or $\varepsilon(\lambda,T)$ we can come around this by applying the integral mean value theorem and use their average instead, as mentioned in Chapter 1. Since both $R(\lambda)$ and $\varepsilon(\lambda,T)$ should be slow-varying functions of the wavelength, there must be certain value $\xi\in [\lambda_{1},\lambda_{2}]$ such that we can express then Equation \ref{eq3.7} as:
		
		\begin{equation}\label{eq3.8}
			N_{meas}(T)= R(\xi) \cdot \varepsilon(\xi,T) \cdot \int_{\lambda_{1}}^{\lambda_{2}} N_{b}(\lambda,T) d\lambda + R(\xi) \cdot [1- \varepsilon(\xi,T)] \cdot \int_{\lambda_{1}}^{\lambda_{2}} N_{b}(\lambda,T_{r}) d\lambda
		\end{equation}	
		
		\begin{equation}\label{eq3.9}
			N_{meas}(T)= R \cdot \bar{\varepsilon} \cdot I_{1}(T) + R \cdot [1- \bar{\varepsilon}] \cdot I_{2}(T_{r})
		\end{equation}
		
		Where $I_{1}(T)$ and $I_{2}(T)$ we can calculate numerically:
		
		\begin{equation}\label{eq3.10}
			I_{1}(T)=\int_{\lambda_{1}}^{\lambda_{2}} N_{b}(\lambda,T) d\lambda
		\end{equation}
		
		\begin{equation}\label{eq3.11}
			I_{2}(T)=\int_{\lambda_{1}}^{\lambda_{2}} N_{b}(\lambda,T_{r}) d\lambda
		\end{equation}
		
		Here $\bar{\varepsilon}=\varepsilon(\xi,T)$ is the averaged emissivity in the wavelength region in which the IR camera is most sensitive. Furthermore, $R(\xi)$ can also be treated as a constant scale factor ($R$) which only depends on the IR sensor. During this study we performed measurements of this factor using an aluminum bucket filled with water at different temperatures and coated with high emissivity electric tape, obtaining consistent values (See scale factors results in Chapter 4).
		
		\begin{figure}[ht!]
			\centering
			\captionsetup{justification=centering,margin=2cm}
			\includegraphics[scale=0.22]{Figures/Chapter03/AngularDistributionSchematics.pdf}
			\caption{Schematics of the angular dependence of the emissive power of a surface in relation to the blackbody}\label{fig3.2}
		\end{figure}
				
		Even though we can assume emissivity as a constant that only depends on the specific material of the surface under investigation, in general, it also depends on the viewing angle of the camera with respect to the normal of the surface. Blackbodies behave like perfect isotropically diffuse emitters, that is, for any surface emitting radiation, the radiance of the emitted radiation is independent of the direction into which it is emitted [8]. However, real surfaces, in addition to the lower emission rate (given by lower emissivity) also show angular dependence in the emissive power (Figure \ref{fig3.2} left).
		
		Usually, depending on the surface, the intensity of the emitted radiation might follow a cosine law with good approximation (Lambertian emitters). However, in general, this is not the case, and in most cases a study must be carried out to estimate the angular dependence of the emissivity [9, 10]. In order to determine whether this is a determining factor in our analysis, an angular study was performed using an aluminum rod filled with hot water placed in the same position as the Petal and, using a strip of high emissivity black tape along the rod's frontal face we were able to measure the variations in temperature due to the viewing angle (Figure \ref{fig3.3}).
				
		\begin{figure}[ht!]
			\centering
			\captionsetup{justification=centering,margin=2cm}
			\includegraphics[scale=0.3]{Figures/Chapter03/AluminumRod.pdf}
			\caption{Aluminum rod used to study the angular dependence of emissivity.}\label{fig3.3}
		\end{figure}
		
		Of course, a combination of all the aforementioned factors leads to the small (but appreciable) differences between several materials in the image despite the fact that they are all at the same temperature. Figure \ref{fig3.4} shows an example of this effects in the emissive power perceived by the IR camera. This image is the ratio of the thermogram collected with the IR camera corresponding to the polished side of the Petal at room temperature (24.4 $^\circ C$ = 297.6 K) with no electronics powered on and no coolant flushed in and a thermogram where all the pixels have the same temperature of 297.6 K. This image clearly shows differences in temperatures among the different materials making visible the shape of the Petal itself. However, under ideal conditions, we shouldn't be able to distinguish any color variation since everything is supposed to be at the same temperature, that is, we should only see pixels with values of 1.
		
		\begin{figure}[ht!]
			\centering
			\captionsetup{justification=centering,margin=2cm}
			\includegraphics[scale=0.5]{Figures/Chapter03/thermo_Temp_201708091701_avg.pdf}
			\caption{Ratio thermogram of the Petal at room temperature. If the IR camera were 100\% accurate we should see all pixels with a ratio value of 1.}\label{fig3.4}
		\end{figure}
		
		In order to obtain an IR image as accurate as possible we must account (correct) for as many of this factors as we can. In the following sections we discuss in more detail the procedures used to estimate the contribution of the most important ones: emissivity, apparent reflected temperature and spectral response function of the camera.
	
		\subsection{Apparent reflected temperature estimation.}\label{section3.1.1}
		
			In order to correctly estimate the emissivity of any surface it is very important to determine first the surface apparent reflected temperature. This is crucial, especially for low emissivity materials where the most important contribution is in fact the reflected radiation.
			A simple method can be used to determine the apparent reflected temperature. This method is described as follows:
			
			\begin{enumerate}[label={\arabic*)}]
				\item Place a piece of crumpled and re-flattened aluminium foil in the same position of the object to be measured.
				\item Set the IR camera emissivity control to 1.00 and distance to 0 m.
				\item Point the IR camera to the target and select a region of interest (ROI).
				\item Measure the apparent surface temperature of the ROI.
				\item Take the measured value as the apparent reflected temperature.
			\end{enumerate}
			
			Aluminum foil has very low emissivity ($\sim$ 0.04) and therefore we are certain that almost all IR radiation coming from the surface is reflected from other sources. The emissivity and distances settings of 1.00 and 0 m respectively are just telling the camera that record everything without applying any further correction. The crumpling is done to allow reflections from several directions, for this reason the ROI selected should be as wide as possible since this effect must be averaged out to obtain an accurate estimate of the apparent reflected temperature.
	
		\subsection{Emissivity estimation. Viewing angle influence.}\label{section3.1.2}
		
			A good illustration of the effect of emissivity in the measurement of the real temperature of a surface is presented in Figure \ref{fig3.5}. This corresponds to the so called Leslie’s Cube. Even though this cube is filled with hot water and all the faces are at the same temperature we can see a huge difference in the temperature measured by the IR camera. This is due to the difference in emissivity of the two front faces. One is painted with a high emissivity black (could be any color) paint and therefore the apparent temperature is closer to the real temperature of the water, while the other face is very reflective (low emissivity) and the apparent temperature is almost entirely due to the reflected component.
			As the main source of discrepancies comes from the differences in emissivity, this is very often the starting point for IR image calibration. Almost all IR cameras can be set to a certain value of emissivity (usually set to 1 by default). If we knew the emissivity value of certain surface we could then enter that value into the IR camera software and the apparent temperature would be close to the real temperature. This is what we would call a “direct” emissivity correction method. It is of course the most simple but also the most uncommon way to correct an IR image for more complicated quantitative studies. On one side, we usually don’t know beforehand the emissivity value of the surface we want to measure and, on the other side, this method would only give us the correct temperature on the surface of known emissivity.
			
			\begin{figure}[ht!]
				\centering
				\captionsetup{justification=centering,margin=2cm}
				\includegraphics[scale=0.6]{Figures/Chapter03/LesliesCube.pdf}
				\includegraphics[scale=0.57]{Figures/Chapter03/LesliesCube2.pdf}
				\caption{Infrared (Left) and visible (Right) images of a Leslie's cube. The cube is filled with hot water and all the faces are at the same temperature. However,  one of the faces is coated with a high emissivity paint (black) and the other has been polished (low emissivity).}\label{fig3.5}
			\end{figure}	
			
			For the Petal's thermal study an equally effective alternative was used: the silicon surface on both sides of the Petal were coated with high emissivity black tape (Figure \ref{fig3.6}). The black tape used has an emissivity of 95\% and therefore the coated surface will behave almost like a blackbody. This is known as the “coating” method and, as the name indicates, consists in coating the surface of interest with a material of known emissivity (ussually high emissivity) and use the direct correction method on the coated surface. 
			For our black taped Petal (“zebra”), setting the IR camera emissivity to the emissivity of the black tape we can measure an apparent temperature very close to the real temperature assuming that both the surface of interest and the black tape on top of it have reached a steady state at the same temperature. Using the IRBIS software a set of over 560 measurement markers (ROIs) were created all over the black tape strips to create a sort of map of temperatures. With this method we could perform relatively accurate temperature measurements on the Petal's silicon surface to be able to compare with FEA simulations (See chapter 4). However, as it can be seen from the picture, we had to cover almost entirely the silicon surface, which would be unacceptable during production stages for quality control. Furthermore, a set of markers was also created for the silicon surface right next to the corresponding black tape marker to study the emissivity behaviour of the silicon using the black tape markers readings as the “real” temperature values.
			
			\bigskip
			There are also some other ways to calculate the emissivity of a surface. One of them consists in estimating the surface’s emissivity by using an alternative temperature measurement as reference. With that purpose, the following method can be used [11, 12]:	
			
			\begin{enumerate}[label={\arabic*)}]
				\item Measure the surface temperature by an alternative method (thermocouple, known emissivity coating).
				\item Set the necessary measurement parameters in the IR camera software and the emissivity to 1.
				\item Point the IR camera to the target and select a region of interest (ROI).
				\item Modify the value of emissivity in the camera software until the apparent temperature in the selected ROI equals the one measured with the alternative method. This is the emissivity of the target surface.
				\item Repeat steps 2) to 4) as many times as necessary to minimize the measurement uncertainty.
			\end{enumerate}	
			
			\begin{figure}[ht!]
				\centering
				\captionsetup{justification=centering,margin=2cm}
				\includegraphics[scale=0.45]{Figures/Chapter03/ZebraPetal.pdf}
				\caption{Unpolished side of the “Zebra” Petal covered with stripes of black electrical tape of high emissivity.}\label{fig3.6}
			\end{figure}
			
			Using this approach a “global” IR image emissivity correction can be applied to get temperature values close to the real ones in the silicon surface. However, as mentioned before, this will only show accurate readings in those areas of the image where the surface of estimated emissivity is. The rest of the elements in the image that have different emissivity values might appear to have a complete different temperature of what they actually have. Even so, if we are not interested in the whole picture but just in one particular material in it, this simple method can be very useful. The problems begin when the surface of interest has relatively low emissivity. In such cases temperature estimation may not be very accurate because low emissivities are hard to estimate and small changes in its value can lead to large variations in the temperature measurements [13, 14]. In other words, the lower the emissivity of the surface, the bigger the systematic uncertainty associated with it.
			Another approach consists in using a “relative” method exploiting the Equation \ref{eq3.9}. Lets imagine that we have two surfaces at the same temperature, then, if we apply Equation \ref{eq3.9} on the one that we know the emissivity value (e.g. black tape) and on the one we want to estimate it from (e.g. silicon surface) then we obtain:
			
			\begin{equation}\label{eq3.12}
				N^{BT}_{meas}(T_{BT})= R \cdot \bar{\varepsilon}_{BT} \cdot I_{1}(T_{BT}) + R \cdot [1- \bar{\varepsilon}_{BT}] \cdot I_{2}(T_{r})
			\end{equation}
			
			\begin{equation}\label{eq3.13}
				N^{Si}_{meas}(T_{BT})= R \cdot \bar{\varepsilon}_{Si} \cdot I_{1}(T_{BT}) + R \cdot [1- \bar{\varepsilon}_{Si}] \cdot I_{2}(T_{r})
			\end{equation}
			
			Note that as the “real” temperature we have selected the apparent temperature of the black tape ($T_{BT}$). We can do so since the black tape has high emissivity. Naturally we know that $N^{BT}_{meas}(T_{BT}) \neq N^{Si}_{meas}(T_{BT})$, however, we can use a small trick here: as the black tape has high emissivity, if we replace $T_{BT}$ in Equation \ref{eq3.12} with the silicon apparent temperature ($T_{Si}$) then $N^{BT}_{meas}(T_{Si}) \equiv N^{Si}_{meas}(T_{BT})$ and after rearrange terms we obtain:
			
			\begin{equation}\label{eq3.14}
				\bar{\varepsilon}_{Si} = \bar{\varepsilon}_{BT} \cdot \frac{I_{1}(T_{Si}) - I_{2}(T_{r})}{I_{1}(T_{BT}) - I_{2}(T_{r})}
			\end{equation}
			
			Note that this expression is invalid if the silicon surface is considerably transmissive, since we have neglected transmissivity for deriving Equation \ref{eq3.9}, and when the real temperature ($T_{BT}$) is close to $T_{r}$ in which case Equation \ref{eq3.14} is undefined.
		
		\subsection{IR camera spectral response.}\label{section3.1.3}
			
			From Equation \ref{eq3.9} it can be seen that if we can accurately estimate emissivity, apparent reflected temperature and the IR camera spectral response function we should be able then to obtain the real temperature of the object for any given output of the camera ($N_{meas}$). 
			In order to obtain the $R$ factor we can simply take the emissive power measurements of a surface of known emissivity, the real temperature and the apparent reflected temperature and then use the same Equation \ref{eq3.9} to derive the $R$ scale factor. As this factor only depends on the IR camera, it can be used later on in the analysis as long as we don’t use a different camera.
			To obtain the scale factor for our analysis we used an aluminum bucket which we filled with hot water and recorded the IR power density on several points of the surface (always at the same depth) coated with black tape as the water temperature went down to test the temperature independence of the results (Figure \ref{fig3.7}). 
						
			\begin{figure}[H]
				\centering
				\captionsetup{justification=centering,margin=2cm}
				\includegraphics[scale=0.12]{Figures/Chapter03/CameraAndBucket.pdf}
				\caption{Experimental setup to calculate the IR camera spectral response function. Measurements were performed placing 8 measurement markers along the horizontal black tape in the camera IRBIS software.}\label{fig3.7}
			\end{figure}	
		
		
	
	\clearpage
	
	% % % Set the style for this file:
\pagestyle{standard}

% % % Beginning of the chapter
\chapter{Results and discussion}

	% % % Set the style for the first page:
	\thispagestyle{chapter-first-page}

	
	\clearpage
	
	% % % Set the style for this file:
\pagestyle{conclusions}

% % % Beginning of the section
\section*{\uppercase{Conclusions}}\label{concl}
	\bigskip
	\bigskip
	With this study, the results of the last two thermal cycles performed on the thermomechanical prototype of the petal are presented. For the first time, a $CO_{2}$ cooling system was used, which allowed us to reach lower temperatures (about -20 $^\circ_{C}$ at inlet pipe). Also, some improvements in the experimental setup related to the insolation of the thermal chamber and data acquisition procedure were adopted. Several PT100 thermocouple sensors were placed on the petal's silicon surface to obtain an alternative temperature reading to compare with the IR measurements. In general, the thermocouples readings and the camera ones (corrected using the black tape method) behave quite similar for most of the sensors. Discrepancies arise possibly related to misplacement of the PT100s are observed for a couple of sensors.\bigskip
	
	Using a set of over 560 measurement areas defined in the IRBIS software, a temperature map of the petal modules was obtained using linear interpolation. The results were compared with the results obtained with the FEA model showing some similitudes.\bigskip
	
	Emissivity of the silicon surface of the unpolished side of the petal has been estimated to be $\varepsilon=0.66\pm0.07$ by means of Equation 3.14 which is in good agreement with the values calculated using the emissivity calculation tool provided by the IR camera software. For the polished side we expect the value to be much less since the surface is highly reflective. This value allows us to globally correct the petal thermograms to show more realistic temperature values (only) on the surface of the silicon.\bigskip
	In addition, the IR camera spectral response scale factor was calculated using an alternative setup consisting in an aluminum bucket filled with water. The results were compared with the ones obtained using a petal thermogram (not powered at room temperature) showing good agreement, as expected for a magnitude only dependent of the IR sensor.\bigskip
	
	Finally, it is shown how the camera's viewing angle has no appreciable influence in the temperature measurements with this particular experimental setup. 
	
	\clearpage
	
	% % % Set the style for this file:
\pagestyle{references}

% % % Beginning of the section
\section*{\uppercase{References}}\label{referen}

	
	\clearpage
	
\end{document}