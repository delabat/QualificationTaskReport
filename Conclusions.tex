% % % Set the style for this file:
\pagestyle{conclusions}

% % % Beginning of the section
\section*{\uppercase{Conclusions}}\label{concl}
	\bigskip
	\bigskip
	The results of the last two thermal cycles performed on the thermomechanical prototype of the petal are presented. For the first time, a $CO_{2}$ cooling system was used, which allowed us to reach lower temperatures (about -20 $^\circ C$ at inlet pipe). Also, some improvements in the experimental setup related to the insolation of the thermal chamber and data acquisition procedure were adopted. Several PT100 thermocouple sensors were placed on the petal’s silicon surface to obtain an alternative temperature reading to compare with the IR measurements. In general, the thermocouples readings and the camera ones (corrected using the black tape method) behave quite similar for most of the sensors. Discrepancies arise possibly related to misplacement of the PT100s are observed for a couple of sensors.
	
	Using a set of over 560 measurement areas defined in the IRBIS software, a temperature map of the petal modules was obtained using linear interpolation. The results were compared with the results obtained with the FEA model showing promising similarities.
	
	Emissivity of the silicon surface of the unpolished side of the petal has been estimated to be $\varepsilon=0.66 \pm 0.07$, assuming no transmissivity, which is in good agreement with the values calculated using the emissivity calculation tool provided by the IR camera software. For the polished side we expect the value to be much less since the surface is highly reflective. This method allows us to globally correct the petal thermograms to show more realistic temperature values (only) on the surface of the silicon. 
	
	In addition, the IR camera spectral response scale factor was calculated using an alternative setup consisting in an aluminum bucket filled with water. The results were compared with the ones obtained using a petal thermogram (not powered at room temperature) showing good agreement, as expected for a magnitude only dependent of the IR sensor.
	
	Finally, it is shown how the camera’s viewing angle has no appreciable influence in the temperature measurements with this particular experimental setup.
	 
	