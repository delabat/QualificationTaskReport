% % % Set the style for this file:
\pagestyle{conclusions}

% % % Beginning of the section
\section*{\uppercase{Conclusion}}\label{concl}
	\bigskip
	\bigskip
	The results of the last two thermal tests (for each side of the petal) performed on the thermo-mechanical petal prototype of the ITk strip sensor for ATLAS Phase II upgrade are presented. For the first time, a CO$_{2}$ cooling system was used, which allowed us to reach lower temperatures (about -20 $^{\circ}$C TRACI data). Also, some improvements in the experimental setup related to the insulation of the thermal chamber and data acquisition procedure were adopted. Several PT100 thermocouple sensors were placed on the petal’s silicon surface to obtain an alternative temperature reading to compare with the IR measurements. In general, the thermocouples readings and the camera ones (corrected using the black tape method) behave quite similar for most of the sensors. Discrepancies possibly related to misplacement of some PT100 sensors were observed.
	
	Additionally, it is shown how the camera’s viewing angle has no appreciable influence in the temperature measurements with this particular experimental setup.
		
	Emissivity of the silicon surface of the unpolished side of the petal has been estimated to be $\varepsilon=0.66 \pm 0.07$, assuming no transmissivity, which is in good agreement with the values calculated using the emissivity calculation tool provided by the IR camera software. This method allows us to globally correct the petal thermograms to show more realistic temperature values (only) on the surface of the silicon. 
		
	Moreover, using a set of over 560 measurement areas defined in the IRBIS software, the temperature maps of the silicon modules in the thermal tests corresponding to both sides of the petal were obtained using linear interpolation. Comparisons with FEA simulations showed fairly good agreement between the model and experimental values specially in areas away from DC-DC converters.
	
	Further investigations are planned to study the thermal response of the petal in different orientations, improving the thermogram correction method by completely covering one side of the petal with high emissivity coating (paint/spray) and also understanding better the CO$_{2}$ properties inside the cooling loop.
	 
	